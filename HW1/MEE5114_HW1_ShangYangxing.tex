\documentclass[titlepage]{article}
\usepackage{graphicx}  % 插入图片
\usepackage{amsmath}  % 输入公式
\usepackage{bm}  % 加粗数学符号
\usepackage{listings}  % 插入代码
\usepackage{xcolor}  % 高亮代码
\usepackage{amssymb}  % 空心粗体

% 设置A4纸
\usepackage[a4paper]{geometry} 

\lstset{numbers=left, %设置行号位置
        numberstyle=\tiny, %设置行号大小
        keywordstyle=\color{blue}, %设置关键字颜色
        commentstyle=\color[cmyk]{1,0,1,0}, %设置注释颜色
        frame=single, %设置边框格式
        escapeinside=``, %逃逸字符(1左面的键),用于显示中文
        breaklines, %自动折行
        extendedchars=false, %解决代码跨页时,章节标题,页眉等汉字不显示的问题
        xleftmargin=2em,xrightmargin=2em, aboveskip=1em, %设置边距
        tabsize=4, %设置tab空格数
        showspaces=false %不显示空格
       }

% 文档信息
\title{\textbf{Advanced Control for Robotics: Homework \#1}}
\author{Shang Yangxing}
\date{\today}

\begin{document}

\maketitle

\section{ODE and Its Simulation}

\subsection{Equation of Pendulum Motions}

\begin{figure}[htbp]
    \centering
    \includegraphics[width=.3\textwidth]{img/pendulum.png}
    \caption{pendulum model}
    \label{fig:pendulum}
\end{figure}

By applying the Newton's law of dynamics, a pendulum with no external force can be formulated as:

\begin{equation}
    ml^2 \ddot{\theta} + ml^2 \alpha \dot{\theta} + mgl \sin{\theta} - T = 0.
\end{equation}

in which,

\quad $m$ is mass of the ball

\quad $l$ is length of the rod

\quad $\alpha$ is the damping constant

\quad $g$ is the gravitational constant

\quad $\theta$ is angle measured between the rod and the vertical axis

\quad $T$ is torque of the joint, which is also the control input $u$

to a system of two first order equation by letting $x_1=\theta$, $x_2=\dot{\theta}$:

\begin{equation}
    \dot{x_1} = x_2, \quad \dot{x_2} = - \frac{g}{l}\sin{x_1} -\alpha x_2 + \frac{T}{ml^2}.
\end{equation}

Written in standard state-space form:

\begin{equation}
    \bm{\dot{x}} = 
    \begin{bmatrix}
        \dot{x_1} \\ \dot{x_2}
    \end{bmatrix}
    =
    \begin{bmatrix}
        x_2 \\ -\frac{g}{l}\sin{x_1} - \alpha x_2
    \end{bmatrix}
    +
    \begin{bmatrix}
        0 \\ \frac{1}{ml^2}
    \end{bmatrix}
    T
    \label{equ:state-space}
\end{equation}

\begin{equation}
    \bm{y} = 
    \begin{bmatrix}
        x_1 \\ x_2
    \end{bmatrix}
    = \bm{x}
\end{equation}

\subsection{Simulation of Pendulum}

When assuming $m=l=1$ with proper unit, equation (\ref{equ:state-space}) can be simplified as:

\begin{equation}
    \begin{bmatrix}
        \dot{x_1} \\ \dot{x_2}
    \end{bmatrix}
    =
    \begin{bmatrix}
        x_2 \\ -g\sin{x_1} - \alpha x_2 + T
    \end{bmatrix}
\end{equation}

according the equation, we code the simulation as following:

\lstinputlisting[language=Python]{1-2 simulation of pendulum.py}

and getting the output as:
\begin{figure}[htbp]
    \centering
    \includegraphics[width=\textwidth]{img/pendulumSim.png}
    \caption{pendulum simulation output}
    \label{fig:pendulumSim}
\end{figure}

\section{Matrix calculus}

\subsection{Tutorial}

\begin{equation}
    \frac{\partial}{\partial X}f(X) = 
    \begin{bmatrix}
        \frac{\partial f(X)}{\partial X_{11}} & \cdots & \frac{\partial f(X)}{\partial X_{1m}} \\
        \vdots & \frac{\partial f(X)}{\partial X_{ij}} & \vdots \\
        \frac{\partial f(X)}{\partial X_{n1}} & \cdots & \frac{\partial f(X)}{\partial X_{nm}}
    \end{bmatrix}
    \label{equ:derivativeWRTMat}
\end{equation}

Derivative of scalar function $f(X)$ can be calculated 
by taking derivatives of the scalar function with respect to 
each entry $X_{ij}$ of the matrix $X$ separately, showing 
as above equation (\ref{equ:derivativeWRTMat}). 

Scalar function $f(X)$ project matrix variable 
$X\in\mathbb{R}^{n\times m}$ to a scalar $y\in\mathbb{R}^{1}$,
so its derivative is the partial derivative, except that 
its results are arranged in form of a matrix, who has the same shape as $X$.

For instance, let's say $X=\begin{bmatrix} X_{11}&X_{12}\\X_{21}&X_{22} \end{bmatrix}$,
$f(X)=\begin{bmatrix}1&1\end{bmatrix}\begin{bmatrix}1\\1\end{bmatrix}
\begin{bmatrix} X_{11}&X_{12}\\X_{21}&X_{22} \end{bmatrix}.$
So $y=f(X)=X_{11}+X_{12}+X_{21}+X_{22}.$ And the partial derivative of $f(X)$ is
\begin{equation}
    \frac{\partial}{\partial X}f(X) = 
    \begin{bmatrix} 
        \frac{\partial f(X)}{\partial X_{11}} & \frac{\partial f(X)}{\partial X_{12}} \\ \\
        \frac{\partial f(X)}{\partial X_{21}} & \frac{\partial f(X)}{\partial X_{22}}
     \end{bmatrix}
     =
     \begin{bmatrix} 
        1 & 1 \\
        1 & 1
     \end{bmatrix}
\end{equation}

\subsection{Derivative of Trace}

\begin{equation}
    \begin{aligned}
        \frac{\partial}{\partial X}tr({AX}) &= \frac{\partial}{\partial X} tr(
        \begin{bmatrix}
            A_{11}X_{11} & \cdots & A_{1m}X_{m1} \\
            \vdots & A_{ij}X_{ji} & \vdots \\
            A_{n1}X_{1n} & \cdots & A_{nm}X_{mn}
        \end{bmatrix}) \\
        &= \frac{\partial}{\partial X} 
        (A_{11}X_{11}+\cdots+A_{ij}X_{ji}+\cdots+A_{nm}X_{mn}) \\
        &=
        \begin{bmatrix}
            A_{11} & \cdots & A_{n1} \\
            \vdots & A_{ji} & \vdots \\
            A_{1m} & \cdots & X_{mn}
        \end{bmatrix}=A^T
    \end{aligned}
\end{equation}

in which, $\frac{\partial}{\partial X_{ij}}
(A_{11}X_{11}+\cdots+A_{ij}X_{ji}+\cdots+A_{nm}X_{mn})=A_{ji}$

\subsection{Derivation}

According to \textit{The Matrix Cookbook} equation (81), we have

\begin{equation}
        \frac{\partial x^TQx}{\partial x} = (Q+Q^T)x
\end{equation}

and we can derive that

\begin{equation}
    \frac{\partial tr(xx^T)}{\partial x} = \frac{\partial}{\partial x}(x_1^2+x_2^2+\cdots+x_n^2)=
    \begin{bmatrix}
        2x_1\\2x_2\\\vdots\\2x_n
    \end{bmatrix}
    =2x
\end{equation}

comprehensive above, we get

\begin{equation}
    \begin{aligned}
        \frac{\partial}{\partial x}f(x)
        &= \frac{\partial x^TQx}{\partial x}+\frac{\partial tr(xx^T)}{\partial x} \\
        &= (Q+Q^T)x + 2x
    \end{aligned}
\end{equation}

\section{Inner product}


\section{Some linear algebra}
\section{Gradient Flow}

\end{document}